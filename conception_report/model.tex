\section{Modèle}

\subsection*{}

\paragraph{}
Nous avons choisi de modéliser le coeur du projet en trois grands packages, comme présenté sur la figure \ref{fig:packages}.

\begin{itemize}
  \item \textbf{World} : contient les éléments nécessaires à la génération et la gestion du monde (carte, cases...)
  \item \textbf{Players} : permet la gestion des joueurs et de leurs unités
  \item \textbf{Engine} : gère l'avancement de la partie et l'enregistrement des actions effectuées par les joueurs. Ce package gère les deux précédents.
\end{itemize}

\begin{figure}[h]
  \centering
  \includegraphics[width=13cm]{schemas/packages.png}
  \caption{Diagramme de packages}
  \label{fig:packages}
\end{figure}

\subsection{Diagramme de classes}

\paragraph{}
La figure \ref{fig:class} présente le diagramme de classe général élaboré pour le projet.
Nous avons fait nos choix de modélisation de façon à avoir le moins de classes possibles, tout en restant dans la simplicité.
Certains choix ont été effectués de façon à introduire les patrons de conception requis pour le projet. Ces choix seront présentés dans la partie suivante.

\paragraph{}
Le diagramme présenté va subir inévitablement des changements lors du développement du projet. Cependant, la structure générale ne devrait pas être amenée à changer. Des diagrammes de séquence seront présentés en partie \ref{diagSequence} afin de conforter nos choix de modélisations. Par soucis de simplicité, les packages ne sont pas représentés.

\begin{figure}[h]
  \centering
  \includegraphics[width=13cm]{schemas/ClassDiagram.png}
  \caption{Diagramme de classes général}
  \label{fig:class}
\end{figure}

\subsection{Patrons de conception}

\paragraph{}
Les différents patrons de conception utilisés dans le projet vont être présentés dans cette partie.

\subsubsection{Génération de carte : Stratégie}

\paragraph{}
Nous avons choisi d'implémenter une Stratégie pour la \textbf{génération de la carte}.
En effet, nous prévoyons d'implémenter plusieurs algorithmes de génération, choisis aléatoirement ou par choix utilisateur parmi la liste non exhaustive suivante :

\begin{itemize}
  \item Génération naïve : il s'agit d'une génération parfaitement aléatoire
  \item Génération d'une île
  \item Génération d'un cratère de volcan
\end{itemize}

Ces algorithmes seront probablement développés en C++ pour acroître la rapidité de leur exécution.
Leur nombre sera défini par le temps restant pour les développer.

\paragraph{}
Le patron Stratégie, présenté dans la figure \ref{fig:strategy}, a été choisi pour sa capacité à exécuter une action de plusieurs façon différentes.

\begin{figure}[h]
  \centering
  \includegraphics[width=13cm]{schemas/dp_strategy.png}
  \caption{Patron de conception Stratégie}
  \label{fig:strategy}
\end{figure}

\subsubsection{Gestion des types de case : Poids-Mouche}

\paragraph{}
Chaque case de la carte étant d'un type parmi 4, nous avons choisi de modéliser les classes représentant les types de cases via un Poids-Mouche.
En effet, beaucoup de cases feront référence au même type, il n'est donc pas nécessaire d'instancier plusieurs fois le même type, ayant à chaque fois les mêmes caractéristiques.
Ce cas de figure correspond parfaitement au Poids-Mouche, bien que le nombre de cases à gérer soit relativement faible. Il est présenté dans la figure \ref{fig:flyweight}.
L'obtention d'un type de case se fait via la classe \emph{TileTypeFactory} (méthode \emph{Get(String type)}). Si une instance du type demandé existe, elle est retournée directement.
Sinon, elle est créée puis retournée.

\paragraph{}
Nous aurions pu également utiliser des classes comportant des informations statiques, ou bien des énumérations pour gagner en simplicité et rapidité.
L'utilisation d'un Poids-Mouche est donc principalement choisie dans un objectif pédagogique.

\begin{figure}[h]
  \centering
  \includegraphics[width=13cm]{schemas/dp_flyweight.png}
  \caption{Patron de conception Poids-Mouche}
  \label{fig:flyweight}
\end{figure}

\subsubsection{Création d'une unité : Fabrique}

\paragraph{}
La création d'une unité pouvant se révéler fastidieuse, nous avons décidé de réaliser une Fabrique d'unités, nommée \emph{Unit Factory}.
Dans la pratique, cette fabrique sera chargée d'initialiser correctement les points de vie d'une unité en fonction de sa race et de mettre à jour sa position sur la carte.
La figure \ref{fig:factory} présente cette fabrique simple.

\begin{figure}[h]
  \centering
  \includegraphics[width=8cm]{schemas/dp_factory.png}
  \caption{Patron de conception Fabrique}
  \label{fig:factory}
\end{figure}

\subsubsection{Initialisation d'une partie : Constructeur}

\paragraph{}
De manière similaire à la fabrique présentée, il est complexe d'initialiser une partie.
Nous avons donc mis en place le patron Constructeur pour gérer cette initialisation, en plusieurs étapes.
Ce constructeur est chargé d'initialiser la carte, de créer les unités et de préparer le démarrage du jeu.

\paragraph{}
Il sera possible de créer une nouvelle partie, ou d'utiliser un fichier de sauvegarde pour relancer une partie interrompue, et ce via les méthodes \emph{New} et \emph{Load} du Constructeur \emph{GameBuilder}.
Le fonctionnement détaillé de l'initialisation d'une partie est détaillé dans la figure \ref{fig:sd_init}, et le schéma du Constructeur est disponible en tant que figure \ref{fig:builder}.

\begin{figure}[h]
  \centering
  \includegraphics[width=8cm]{schemas/dp_builder.png}
  \caption{Patron de conception : Constructeur}
  \label{fig:builder}
\end{figure}

\subsubsection{Annulation d'une action : Commande / Memento}

\label{command}

\paragraph{}
Enfin, nous avons utilisé le patron Commande, associé au patron Memento et présenté dans la figure \ref{fig:command}.
L'objectif de cette structure est la \textbf{conservation des actions effectuées} par les joueurs au cours de la partie, afin d'être capable d'inverser ces actions (quand un mode triche est activé).
Les deux actions possibles (attaque et mouvement) contiennent les informations nécessaires à leur inversion, et la liste des actions effectuées sera stockée dans la classe \emph{Game}.

\paragraph{}
Le fonctionnement de ce patron étant assez subtil, un diagramme de séquence est disponible dans le cas d'une bataille en figure \ref{fig:sd_battle}.

\begin{figure}[h]
  \centering
  \includegraphics[width=13cm]{schemas/dp_command.png}
  \caption{Patron de conception : Commande / Memento}
  \label{fig:command}
\end{figure}

\subsection{Diagrammes de séquence}
\label{diagSequence}
\subsubsection{Initialisation d'une nouvelle partie}

\paragraph{}
Le diagramme \ref{fig:sd_init} présente l'initialisation du jeu dans le cas d'une nouvelle partie.
Dans un premier temps, les deux joueurs définissent leur race (instantiations de la classe \emph{Player}).
Ensuite, le \emph{GameBuider} est appelé et provoque la génération de la carte et l'initialisation des différentes classes utilisées.
Les unités sont également instanciées et positionnées sur la carte.

\begin{figure}[h]
  \centering
  \includegraphics[width=13cm]{schemas/sd_init.png}
  \caption{Diagramme de séquence de l'initialisation du jeu}
  \label{fig:sd_init}
\end{figure}

\subsubsection{Bataille}

\paragraph{}
L'exécution d'une bataille est présentée dans le diagramme de séquence \ref{fig:sd_battle}.
Dans un premier temps, le joueur courant sélectionne une unité qu'il souhaite actionner (ici, \emph{unitA}).
Une liste \textbf{d'actions possibles} est alors générée. Ces actions sont soit des mouvements, soit des attaques en fonction des unités adverses présentes (ou pas) sur les cases voisines.
Le joueur peut alors choisir une action à exécuter dans la liste renvoyée.

\paragraph{}
Une fois l'action exécutée via le patron de conception Commande présenté en \ref{command}, la classe \emph{Game} enregistre l'action effectuée dans une pile et vérifie que le jeu n'est pas encore terminé.
Par la suite, dans le cas où il n'y a plus aucun défenseur sur la case ciblée, une action \emph{Move} est automatiquement créée, exécutée et ajoutée à la pile pour déplacer l'unité victorieuse sur la carte sans consommation de points de mouvement.

\begin{figure}[h]
  \centering
  \includegraphics[width=13cm]{schemas/sd_battle.png}
  \caption{Diagramme de séquence d'une bataille}
  \label{fig:sd_battle}
\end{figure}
