\section{Modèle}

\subsection{Diagramme de classes}

\paragraph{}
La figure \ref{fig:class} présente le diagramme de classe général élaboré pour le projet.
Nous avons fait nos choix de modélisation de façon à avoir le moins de classes possibles, tout en restant dans la simplicité.
Certains choix ont été effectués de façon à introduire les patrons de conception requis pour le projet. Ces choix seront présentés dans la partie suivante.

\paragraph{}
Le diagramme présenté va subir inévitablement des changements lors du développement du projet. Cependant, la structure générale ne devrait pas être amenée à changer. Des diagrammes de séquence seront présentés en partie \ref{diagSequence} afin de conforter nos choix de modélisations.

\begin{figure}[h]
  \centering
  \includegraphics[width=13cm]{schemas/ClassDiagram.png}
  \caption{Diagramme de classes général}
  \label{fig:class}
\end{figure}

\subsection{Patrons de conception}

\paragraph{}
Les différents patrons de conception utilisés dans le projet vont être présentés dans cette partie.

\subsubsection{Génération de carte : Stratégie}

\paragraph{}
Nous avons choisi d'implémenter une Stratégie pour la \textbf{génération de la carte}.
En effet, nous prévoyons d'implémenter plusieurs algorithmes de génération, choisis aléatoirement ou par choix utilisateur parmi la liste non exhaustive suivante :

\begin{itemize}
  \item Génération naïve : il s'agit d'une génération parfaitement aléatoire
  \item Génération d'une île
  \item Génération d'un cratère de volcan
\end{itemize}

Ces algorithmes seront probablement développés en C++ pour acroître la rapidité de leur exécution.
Leur nombre sera défini par le temps restant pour les développer.

\paragraph{}
Le patron Stratégie, présenté dans la figure \ref{fig:strategy}, a été choisi pour sa capaciter à exécuter une action de plusieurs façon différentes.

\begin{figure}[h]
  \centering
  \includegraphics[width=13cm]{schemas/dp_strategy.png}
  \caption{Patron de conception Stratégie}
  \label{fig:strategy}
\end{figure}

\subsubsection{Gestion des types de case : Poids-Mouche}

\paragraph{}
Chaque case de la carte étant d'un type parmi 4, nous avons choisi de modéliser les classes représentant les types de cases via un Poids-Mouche.
En effet, beaucoup de cases feront référence au même type, il n'est donc pas nécessaire d'instancier plusieurs fois le même type, ayant à chaque fois les mêmes caractéristiques.
Ce cas de figure correspond parfaitement au Poids-Mouche, bien que le nombre de cases à gérer soit relativement faible. Il est présenté dans la figure \ref{fig:flyweight}.
L'obtention d'un type de case se fait via la classe \emph{TileTypeFactory} (méthode \emph{Get(String type)}). Si une instance du type demandé existe, elle est retournée directement.
Sinon, elle est créée puis retournée.

\paragraph{}
Nous aurions pu également utiliser des classes comportant des informations statiques, ou bien des énumérations pour gagner en simplicité et rapidité.
L'utilisation d'un Poids-Mouche est donc principalement choisie dans un objectif pédagogique.

\begin{figure}[h]
  \centering
  \includegraphics[width=13cm]{schemas/dp_flyweight.png}
  \caption{Patron de conception Poids-Mouche}
  \label{fig:flyweight}
\end{figure}

\subsubsection{Création d'une unité : Fabrique}
\begin{figure}[h]
  \centering
  \includegraphics[width=13cm]{schemas/dp_factory.png}
  \caption{Patron de conception Fabrique}
  \label{fig:factory}
\end{figure}

\subsubsection{Initialisation d'une partie : Constructeur}
\begin{figure}[h]
  \centering
  \includegraphics[width=13cm]{schemas/dp_builder.png}
  \caption{Patron de conception : Constructeur}
  \label{fig:builder}
\end{figure}

\subsubsection{Annulation d'une action : Commande / Memento}
\begin{figure}[h]
  \centering
  \includegraphics[width=13cm]{schemas/dp_command.png}
  \caption{Patron de conception : Commande / Memento}
  \label{fig:command}
\end{figure}

\subsection{Diagramme de séquence}
\label{diagSequence}
\subsubsection{Initialisation}

\begin{figure}[h]
  \centering
  \includegraphics[width=13cm]{schemas/sd_init.png}
  \caption{Diagramme de séquence de l'initialisation du jeu}
  \label{fig:sd_init}
\end{figure}

\subsubsection{Bataille}

\begin{figure}[h]
  \centering
  \includegraphics[width=13cm]{schemas/sd_battle.png}
  \caption{Diagramme de séquence d'une bataille}
  \label{fig:sd_battle}
\end{figure}
