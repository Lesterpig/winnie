\section{Présentation du jeu}

\subsection{Concepts}

\paragraph{}
Le jeu Small World oppose deux joueurs de deux races différentes aux choix parmi les \textbf{Humains}, les \textbf{Elfes} et les \textbf{Orcs}.
Chaque joueur possède un certain nombre d'unités ayant les mêmes caractéristiques. Chaque unité possède un certain nombre de \textbf{points de vie}, de \textbf{points d'attaque} et de \textbf{points de défense} en fonction de sa race.
Le tableau \ref{fig:caracteristiques} présente ces caractéristiques.

\begin{table}[h!]
  \centering
  \begin{tabular}{|l|l|l|l|}
    \hline
    Race&Points de vie&Attaque&Défense\\
    \hline
    Humain&15&6&3\\
    \hline
    Elfe&12&4&3\\
    \hline
    Orc&17&5&2\\
    \hline
  \end{tabular}
  \caption{Caractéristiques de base des différentes races du jeu}
  \label{fig:caracteristiques}
\end{table}

\paragraph{}
Le nombre de points d'attaque et de défense sont altérés par le \textbf{ratio de points de vie restants} : plus une unité est blessée, et plus ses points d'attaque et de défense diminuent. Les formules utilisées sont les suivantes :

\begin{displaymath}
  PointsAttaque = \lceil \frac{PointsDeVie}{VieMax} AttaqueMax \rceil
\end{displaymath}
\begin{displaymath}
  PointsDefense = \lceil \frac{PointsDeVie}{VieMax} DefenseMax \rceil
\end{displaymath}

En aucun cas une unité ne peut regagner des points de vie, d'attaque ou de défense. Les unités ayant moins de 1 point de vie sont éliminées de la partie.

\paragraph{}
Les unités peuvent se déplacer sur une carte en deux dimensions composés de blocs carrés.
Il existe quatre types de blocs : \textbf{eau, plaine, forêt et montagne}. Le nombre de \textbf{points de victoire} obtenus par une unité dépend de sa race et du type de bloc sur lequel elle est positionnée. % TODO tableau
 % TODO points de mouvement
